\subsection{Compound interest and rates}

\begin{defn}[Accumulation functions]
The \emph{accumulation (amount) function} $A(t)$ is the value of an investment $P = A(0)$ at time $t$. The \emph{accumulation (factor) function} $a(t)$ is the value at time $t$ of an initial investment of 1,
\begin{equation}
A(t) = A(0)a(t)\quad\longleftrightarrow\quad a(t) = \frac{A(t)}{A(0)}.
\end{equation}
\end{defn}

\begin{example}
If an annual interest rate $i$ is fixed, then after $n$ whole years,
\begin{equation}
a(n) = (1 + i)^n\quad\text{and}\quad A(n) = P(1 + i)^n,
\end{equation}
where $P = A(0)$ is the principal investment.
\end{example}

In practice, we often use a smaller compounding period, such as a quarter, month, or day. It is often helpful to be able to convert to another compounding period.

\begin{defn}[Equivalence of rates]
Two interest rates are \emph{equivalent} if their accumulation functions are equal at any time for which a whole number of compounding periods have taken place for both rates. For a given interest rate, the \emph{effective (annual) interest rate} is the equivalent annual interest rate:
\begin{equation}
1 + i_{\text{eff}} = a(1)\implies i_{\text{eff}} = a(1) - 1,
\end{equation}
where $a(1)$ is calculated using the given interest rate.
\end{defn}

\begin{example}
A quarterly interest rate of $3\%$ yields an effective annual interest rate
\begin{equation}
i_{\text{eff}} = (1.03)^4 - 1\approx 12.55\%.
\end{equation}
\end{example}

Carrying out the same calculation, the effective annual interest rate for a rate $r$ compounded $m$ times a year (at evenly spaced time intervals) is
\begin{equation}
i_{\text{eff}} = (1 + r)^m - 1.
\end{equation}

\begin{defn}[Nominal annual interest rate]
Given an interest rate $r$ compounded $m$ times a year, the \emph{nominal annual interest rate} is $mr$. When $m = 12$, so that compounding is monthly, $r$ is the \emph{periodic rate} and the nominal annual interest rate is the \emph{annual percentage rate (APR)}.
\end{defn}

We denote by $i^{(m)}$ the nominal annual interest rate for which compounding $m$ times a year yields an effective annual interest rate of $i$. Thus
\begin{equation}
i = \left(1 + \frac{i^{(m)}}{m}\right)^m - 1\implies i^{(m)} = m[(1 + i)^{1/m} - 1].
\end{equation}
In the limit $m\to\infty$ of \emph{continuous compounding}, we obtain the \emph{force of interest}
\begin{equation}
\begin{split}
\delta &= \lim_{m\to\infty} i^{(m)} \\
&= \lim_{m\to\infty} m[(1 + i)^{1/m} - 1] \\
&= \lim_{t\to 0}\frac{(1 + i)^t - 1}{t} \\
&= \frac{d}{dt}(1 + i)^t\text{ at }t = 0 \\
&= \ln(1 + i).
\end{split}
\end{equation}