\section{Quadratics}

In this section, we review the main ideas in the theory of quadratics in one variable. A \emph{quadratic in $X$} is an expression of the form $aX^2 + bX + c$, where $a$, $b$, and $c$ are constants (or at least independent of $X$) and $a\neq 0$. A \emph{root} of the quadratic expression $aX^2 + bX + c$ is a value $r$ for which $ar^2 + br + c = 0$.

\subsection{Finding roots by factoring}

One way that quadratic expressions arise is as a product of two linear expressions,
\begin{equation*}
(X - 2)(X + 3) = X(X + 3) - 2(X + 3) = X^2 + 3X - 2X - 6 = X^2 + X - 6. 
\end{equation*}
For a given quadratic, if we can find linear factors, identifying roots becomes straightforward.

\begin{example}
Find the roots of $X^2 + X - 6$. 
\end{example}
\begin{solution}
Since this quadratic is equivalent to $(X - 2)(X + 3)$, the values of $X$ which make the expression evaluate to $0$ are those values $r$ for which either $r - 2 = 0$ or $r + 3 = 0$. Hence the roots of $X^2 + X - 6$ are $2$ and $-3$.
\end{solution}