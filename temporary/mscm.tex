\documentclass{article}
\usepackage{mathtools,amssymb}
\usepackage{float,graphicx}
\usepackage{nopageno}
\usepackage[letterpaper]{geometry}

\setlength{\parindent}{0in}


\begin{document}

\begin{enumerate}
\item \underline{\hspace{3in}}\vspace{1cm}
\item \underline{\hspace{3in}}\vspace{1cm}
\item \underline{\hspace{3in}}\vspace{1cm}
\item \underline{\hspace{3in}}\vspace{1cm}
\item \underline{\hspace{3in}}\vspace{1cm}
\item \underline{\hspace{3in}}\vspace{1cm}
\item \underline{\hspace{3in}}\vspace{1cm}
\item \underline{\hspace{3in}}\vspace{1cm}
\item In regular pentagon $ABCDE$, diagonals $\overline{AC}$ and $\overline{AD}$ meet diagonal $\overline{BD}$ at points $X$ and $Y$, respectively. Given that $AB = 1$, what is the length of $\overline{XY}$? Express your answer as a common fraction in simplest radical form. [\emph{We found many of the angles in the diagram last week, and these will be helpful for finding isosceles triangles and similar triangles.}]
\vspace{1cm}
\item Tosh rolls two standard six-sided dice while Brant rolls one standard six-sided die. What is the probability that the larger of Tosh's rolls is (strictly) greater than Brant's roll? Express your answer as a common fraction.
\end{enumerate}


\newpage

\begin{enumerate}
\item $1/3$\vspace{1cm}
\item $3/2$\vspace{1cm}
\item $9/2$\vspace{1cm}
\item $2\sqrt{5}$ (\emph{MATHCOUNTS 2016: State Countdown}) \vspace{1cm}
\item $3\sqrt{2}$\vspace{1cm}
\item $2$\vspace{1cm}
\item $1/2$\vspace{1cm}
\item 81 (\emph{MATHCOUNTS 2013: National Sprint \#14}) \vspace{1cm}
\item In regular pentagon $ABCDE$, diagonals $\overline{AC}$ and $\overline{AD}$ meet diagonal $\overline{BD}$ at points $X$ and $Y$, respectively. Given that $AB = 1$, what is the length of $\overline{XY}$? Express your answer as a common fraction in simplest radical form. $\boxed{\dfrac{3 - \sqrt{5}}{2}}$
\vspace{1cm}
\item Tosh rolls two standard six-sided dice while Brant rolls one standard six-sided die. What is the probability that the larger of Tosh's rolls is (strictly) greater than Brant's roll? Express your answer as a common fraction. $\boxed{125/216}$
\end{enumerate}



\end{document}

