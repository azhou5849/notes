\documentclass{article}
\usepackage{mathtools,amssymb}
\usepackage{float,graphicx}
\usepackage{nopageno}
\usepackage[letterpaper]{geometry}

\setlength{\parindent}{0in}
\setlength{\parskip}{3pt plus 1pt minus 1pt}


\begin{document}

\section*{Geometric Probability}

\begin{enumerate}
\item A real number $u$ between 0 and 5 is chosen at random. What is the probability that $u$ is within $1/5$ of an integer?\vspace{3cm}
\item (2004 National Team) A point $P$ is randomly placed in the interior of right triangle $ABC$ with right angle at $C$. What is the probability that the area of triangle $PBC$ is less than half of the area of triangle $ABC$?\vspace{3cm}
\item (2004 National Target) Given that $a$ and $b$ are real numbers chosen independent at random so that $-3\leq a\leq 1$ and $-2\leq b\leq 4$, what is the probability that the product $a\cdot b$ is positive?\vspace{3cm}
\item Randomly picking two points independently on the circumference of a unit circle, what is the probability that the straight-line distance between the two points is shorter than 1?\vspace{3cm}
\item If we choose two numbers between 0 and 2 independently at random, what is the probability that their sum is smaller than 1?
\end{enumerate} 


% \newpage

% \section*{Repeated Elements}

% \begin{enumerate}
% \item If all the letters of the word $SYZYGY$\footnote{``In astronomy, a roughly straight-line configuration of three or more celestial bodies.''} are used, in how many different ways can the six letters be arranged in a six-letter string?\vspace{4cm}
% \item In how many different ways can the letters in the word $PEOPLE$ be scrambled, including the original spelling $PEOPLE$?\vspace{4cm}
% \item (2006 State Sprint Problem 28) Derek's phone number, 336-7624, has the property that the three-digit prefix, 336, equals the product of the last four digits, $7\times 6\times 2\times 4$. How many seven-digit phone numbers beginning with 336 have this property?
% \end{enumerate}


\newpage

\section*{Extensions}
\vspace{1cm}
\begin{enumerate}
\item \underline{\hspace{3in}}\vspace{1cm}
\item \underline{\hspace{3in}} [common fraction, in terms of $\pi$]\vspace{1cm}
\item \underline{\hspace{3in}}\vspace{1cm}
\item \underline{\hspace{3in}} [common fraction, in terms of $\pi$]\vspace{1cm}
\item \underline{\hspace{3in}} (2009 National Countdown)\vspace{1cm}
\item \underline{\hspace{3in}} [simplest radical form]\vspace{1cm}
\item \underline{\hspace{3in}} [common fraction]\vspace{1cm}
\item \underline{\hspace{3in}} [common fraction, in terms of $\pi$]
\end{enumerate}


\newpage

\section*{Extra Problems ($\star$)}

\begin{enumerate}
\item Define a sequence $a_0, a_1, a_2, \ldots$ by $a_1 = 6$ and
\begin{equation*}
a_n = \frac{-1}{a_{n - 1} + 1}
\end{equation*}
for all integers $n\geq 2$. What is the value of $a_{2025}$? Express your answer as a common fraction.\vspace{3cm}
\item Three real numbers are chosen independently and uniformly at random between 0 and 1. What is the probability that these three real numbers can be the side lengths of a triangle? Express your answer as a common fraction.\vspace{3cm}
\item Let $ABC$ be a triangle with $AB = 12$, $\angle A = 15^{\circ}$, and $\angle B = 30^{\circ}$. Find the length of $BC$, expressing your answer in simplest radical form.\vspace{3cm}
\item For each positive integer $n$, let $d(n)$ denote the number of positive integer divisors of $n$. For example, $d(6) = 4$ and $d(16) = 5$. Given that
\begin{equation*}
\left\lfloor\frac{100}{1}\right\rfloor + \left\lfloor\frac{100}{2}\right\rfloor + \left\lfloor\frac{100}{3}\right\rfloor + \cdots + \left\lfloor\frac{100}{10}\right\rfloor = 291,
\end{equation*}
what is the value of
\begin{equation*}
d(1) + d(2) + d(3) + \cdots + d(99) + d(100)?
\end{equation*}
\end{enumerate}


\end{document}