\section{Quadratics}

A \emph{quadratic (polynomial) in $X$} is an expression of the form $aX^2 + bX + c$, where $a,b,c$ are independent of $X$ and $a\neq 0$. Collectively $a,b,c$ are the \emph{coefficients} of the quadratic while $aX^2, bX, c$ are the \emph{terms} of the quadratic.
\begin{itemize}
\item The term $c$ is called the \emph{constant term}, as it does not depend on $X$, or the \emph{degree-0 term} (as it can be written as $cX^{\color{red}0}$).
\item The term $bX$ is called the \emph{linear term} or the \emph{degree-1 term} (as it can be written as $bX^{\color{red}1}$), and there are analogous names for the coefficient $b$.
\item The term $aX^2$ is called the \emph{quadratic term} or the \emph{degree-2 term}, and there are analogous names for the coefficient $a$. Since this term contains the highest power of $X$ in the whole expression, it is also called the \emph{leading term}, with $a$ correspondingly called the \emph{leading coefficient}.
\end{itemize}
A \emph{root} of the quadratic expression $aX^2 + bX + c$ is a value $r$ for which $ar^2 + br + c = 0$.

\subsection{Finding roots by factoring}

One way that quadratic expressions arise is as a product of two linear expressions,
\begin{equation*}
(X - 2)(X + 3) = X(X + 3) - 2(X + 3) = X^2 + 3X - 2X - 6 = X^2 + X - 6. 
\end{equation*}
For a given quadratic, if we can find linear factors, identifying roots becomes straightforward.

\begin{example}
Find the roots of $X^2 + X - 6$. 
\end{example}
\begin{solution}
Let $r$ be a root, so by definition, we need $r^2 + r - 6 = 0$. By our calculation above, the left hand side is equal to $(r - 2)(r + 3)$. For a product of two (or more) factors to be equal to $0$, at least one of them must be $0$. Therefore, any root $r$ must satisfy $r - 2 = 0$ or $r + 3 = 0$, and if $r$ satisfies at least one of these two equations, it is a root. Hence the roots of $X^2 + X - 6$ are $2$ and $-3$.
\end{solution}

\begin{example}
Find the roots of $X^2 - 8X + 12$.
\end{example}
\begin{solution}
This time, we need to find a factorisation of $X^2 - 8X + 12$ first. A reasonable guess is that the factorisation has the form $(X + A)(X + B)$, where $A$ and $B$ are constants to be determined. Expanding,
\begin{align*}
(X + A)(X + B) &= X(X + B) + A(X + B) \\
&= X^2 + BX + AX + AB \\
&= X^2 + (A + B)X + AB.
\end{align*}
The coefficients must match, 
\end{solution}