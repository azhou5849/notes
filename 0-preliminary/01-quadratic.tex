\section{Quadratics}

A \emph{quadratic (polynomial) in $X$} is an expression of the form $aX^2 + bX + c$, where $a,b,c$ are independent of $X$ and $a\neq 0$. Collectively $a,b,c$ are the \emph{coefficients} of the quadratic while $aX^2, bX, c$ are the \emph{terms} of the quadratic.

\subsection{Factoring quadratics whose leading coefficient is 1}

Some quadratics arise as a product of two linear expressions, such as
\begin{equation*}
(X - 2)(X + 3) = X(X + 3) - 2(X + 3) = X^2 + 3X - 2X - 6 = X^2 + X - 6. 
\end{equation*}
\emph{Factoring} refers to the reverse process of finding, for a given quadratic, two linear expressions which multiply to that quadratic. Here we focus on factoring \underline{over the integers}.

\begin{example}
Factor $X^2 + 8X + 12$.
\end{example}
\begin{solution}
A reasonable guess is that a factorization has the form $(X + A)(X + B)$. Expanding,
\begin{equation*}
(X + A)(X + B) = X(X + B) + A(X + B) = X^2 + (A + B)X + AB.
\end{equation*}
Matching coefficients, we want $A + B = 8$ and $AB = 12$. Listing out pairs of integers which multiply to 12, or pairs of integers which sum to 8, we find that if $A$ and $B$ are $2$ and $6$ (in either order), both equations hold. Therefore, $X^2 + 8X + 12 = \boxed{(X + 2)(X + 6)}$.
\end{solution}

\begin{example}
Factor $X^2 - 10X + 21$.
\end{example}
\begin{solution}
Setting up a factorization $(X + A)(X + B)$ as before, this time $A + B = -10$ and $AB = 21$. We find $A$ and $B$ are $-3$ and $-7$ in some order, so $X^2 - 10X + 21 = \boxed{(X - 3)(X - 7)}$.
\end{solution}

To make guess-and-check easier, we can start with some sign analysis. This allows us to narrow our search to \underline{positive} integers instead of all integers.

\begin{example}
Factor $X^2 - 19X + 48$.
\end{example}
\begin{solution}
Since the constant term $48$ is positive, we know that the constants of the factors are both positive or both negative. Since the linear coefficient $-19$ is negative, those constants have to be negative. Therefore, we can set up a factorization of the form
\begin{equation*}
(X - A)(X - B) = X^2 - (A + B)X + AB.
\end{equation*}
We need two positive integers whose sum is 19 and whose product is 48. These turn out to be 3 and 16, so $X^2 - 19X + 48 = \boxed{(X - 3)(X - 16)}$.
\end{solution}

\begin{example}
Factor $X^2 + 7X - 44$.
\end{example}
\begin{solution}
Since the constant term $-44$ is negative, we can set up a factorization of the form
\begin{equation*}
(X + A)(X - B) = X^2 + (A - B)X - AB.
\end{equation*}
We need two positive integers whose product is 44 and whose difference is 7. These turn out to be 4 and 11. As $A - B = 7$, we require $A = 11$ and $B = 4$, so $X^2 + 7X - 44 = \boxed{(X + 11)(X - 4)}$.
\end{solution}


\subsection{Factoring quadratics whose leading coefficient is not 1}

So far, all of the examples we considered are \emph{monic}, meaning the leading coefficient (coefficient of $X^2$) is 1. When the leading coefficient is not 1, the task becomes more challenging.

\begin{example}
Factor $3X^2 + 10X + 8$.
\end{example}
\begin{proof}

\end{proof}


\subsection{Finding roots by factoring}

One way that quadratic expressions arise is as a product of two linear expressions,

For a given quadratic, if we can find linear factors, identifying roots becomes straightforward.

\begin{example}
Find the roots of $X^2 + X - 6$. 
\end{example}
\begin{solution}
Let $r$ be a root, so by definition, we need $r^2 + r - 6 = 0$. By our calculation above, the left hand side is equal to $(r - 2)(r + 3)$. For a product of two (or more) factors to be equal to $0$, at least one of them must be $0$. Therefore, any root $r$ must satisfy $r - 2 = 0$ or $r + 3 = 0$, and if $r$ satisfies at least one of these two equations, it is a root. Hence the roots of $X^2 + X - 6$ are $2$ and $-3$.
\end{solution}

\begin{example}
Find the roots of $X^2 - 8X + 12$.
\end{example}
\begin{solution}
This time, we need to find a factorisation of $X^2 - 8X + 12$ first. A reasonable guess is that the factorisation has the form $(X + A)(X + B)$, where $A$ and $B$ are constants to be determined. Expanding,
\begin{align*}
(X + A)(X + B) &= X(X + B) + A(X + B) \\
&= X^2 + BX + AX + AB \\
&= X^2 + (A + B)X + AB.
\end{align*}
The coefficients must match, 
\end{solution}