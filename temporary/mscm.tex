\documentclass{article}
\usepackage{mathtools,amssymb}
\usepackage{float,graphicx}
\usepackage{nopageno}
\usepackage[letterpaper]{geometry}

\setlength{\parindent}{0in}


\begin{document}

\begin{enumerate}
\item \underline{\hspace{3in}} (in terms of $\pi$)\vspace{1cm}
\item \underline{\hspace{3in}} (common fraction)\vspace{1cm}
\item \underline{\hspace{3in}} (in terms of $\pi$)\vspace{1cm}
\item \underline{\hspace{3in}} (common fraction)\vspace{1cm}
\item \underline{\hspace{3in}}\vspace{1cm}
\item \underline{\hspace{3in}}\vspace{1cm}
\item \underline{\hspace{3in}} (simplest radical form)\vspace{1cm}
\item \underline{\hspace{3in}} (in terms of $\pi$)\vspace{1cm}
\item Let $ABCD$ be a quadrilateral inscribed in a circle (with the vertices in that order), and suppose that $\angle BAD = 66^{\circ}$, $\angle CDB = 23^{\circ}$, and $\angle BDA = 61^{\circ}$. If $E$ is the intersection of diagonals $\overline{AC}$ and $\overline{BD}$, calculate $\angle AED$ in degrees.
\vspace{1cm}
\item Let $ABC$ be a triangle with $AB = 13$, $BC = 14$, and $CA = 15$. Point $I$ is the center of the (inscribed) circle tangent to all three sides of $ABC$. Compute $CI$, expressing your answer in simplest radical form.
\end{enumerate}


\newpage

\begin{enumerate}
\item $8\pi$\vspace{1cm}
\item $9/4$\vspace{1cm}
\item $17\pi$\vspace{1cm}
\item $1/72$ (\emph{MATHCOUNTS 2013: Chapter Sprint})\vspace{1cm}
\item $3$\vspace{1cm}
\item $6$ (\emph{MATHCOUNTS 2013: National Sprint})\vspace{1cm}
\item $16 + 2\sqrt{3}$\vspace{1cm}
\item $5\pi$\vspace{1cm}
\item Let $ABCD$ be a quadrilateral inscribed in a circle (with the vertices in that order), and suppose that $\angle BAD = 66^{\circ}$, $\angle CDB = 23^{\circ}$, and $\angle BDA = 61^{\circ}$. If $E$ is the intersection of diagonals $\overline{AC}$ and $\overline{BD}$, calculate $\angle AED$ in degrees. $\boxed{76}$
\vspace{1cm}
\item Let $ABC$ be a triangle with $AB = 13$, $BC = 14$, and $CA = 15$. Point $I$ is the center of the (inscribed) circle tangent to all three sides of $ABC$. Compute $CI$, expressing your answer in simplest radical form. $\boxed{4\sqrt{5}}$
\end{enumerate}



\end{document}

