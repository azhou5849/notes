\documentclass{article}
\usepackage{mathtools,amssymb}
\usepackage{float,graphicx}
\usepackage{nopageno}
\usepackage[letterpaper]{geometry}

\setlength{\parindent}{0in}


\begin{document}

\section*{Permutations}

\textbf{Definition:} The \emph{factorial} of a non-negative integer $n$ is defined recursively by
\begin{equation*}
0! = 1\quad\text{and}\quad n! = n\cdot (n - 1)!\text{ for all }n\geq 1.
\end{equation*}
When $n$ is a positive integer, $n!$ is the product of all positive integers less than or equal to $n$, and it gives the number of orders in which $n$ distinct items can be listed. The first several values are
\begin{align*}
0! &= 1, \\
1! &= 1\cdot 0! = 1\cdot 1 = 1, \\
2! &= 2\cdot 1! = 2\cdot 1 = 2, \\
3! &= 3\cdot 2! = 3\cdot 2 = 6, \\
4! &= 4\cdot 3! = 4\cdot 6 = 24, \\
5! &= 5\cdot 4! = 5\cdot 24 = 120, \\
6! &= 6\cdot 5! = 6\cdot 120 = 720.
\end{align*}
\begin{enumerate}
\item The library is giving one book to each student for free. Three friends show up at the library and find that there are 4 different books available, with only one copy left for each of the books. In how many ways can the friends choose their books?\vspace{2cm}
\item How many positive 3-digit integers have 3 distinct digits?\vspace{2cm}
\item How many 4-digit \underline{odd} integers greater than 6000 can be formed from the digits $\{0, 1, 3, 5, 6, 8\}$ if no digit may be used more than once?\vspace{2cm}
\item How many different 4-letter strings can be generated by using each of the letters $A$, $O$, $P$, and $S$ exactly once? (A \emph{string} in this context is a sequence of characters, such as $SOAP$. It need not be an actual English word, so for example, $SAOP$ would also be a valid string.)
\end{enumerate} 


\newpage

\section*{Repeated Elements}

\begin{enumerate}
\item If all the letters of the word $SYZYGY$\footnote{definition} are used, in how many different ways can the six letters be arranged in a six-letter string?\vspace{3cm}
\item (2002 School Sprint Problem \#22) How many odd whole numbers are factors of 180?\vspace{2cm}
\item (2012 National Sprint Problem \#28) How many whole numbers $n$, such that $100\leq n\leq 1000$, have the same number of odd factors as even factors?\vspace{2cm}
\item (2006 State Team) Emma plays with her square unit tiles by arranging all of them into different shaped rectangular figures. (For example, a $5\times 7$ rectangle would use 35 tiles and would be considered the same rectangle as a $7\times 5$ rectangle.) Emma can form exactly ten different such rectangular figures that each use all of her tiles. What is the least number of tiles Emma could have?\vspace{2cm}
\end{enumerate}


\newpage

\section*{Extensions}
\vspace{1cm}
\begin{enumerate}
\item \underline{\hspace{3in}}\vspace{1cm}
\item \underline{\hspace{3in}}\vspace{1cm}
\item \underline{\hspace{3in}}\vspace{1cm}
\item \underline{\hspace{3in}}\vspace{1cm}
\item \underline{\hspace{3in}}\vspace{1cm}
\item \underline{\hspace{3in}}\vspace{1cm}
\item \underline{\hspace{3in}} (skippable, or feel free to be silly)\vspace{1cm}
\item \underline{\hspace{3in}}
\end{enumerate}


\newpage

\section*{Extra Problems}

\begin{enumerate}
\item A sequence of numbers has the property that the sum of the first $n$ terms is given by $n^3 + n + 1$. What is the 100th term of the sequence?\vspace{3cm}
\item An isosceles trapezoid has bases of lengths 6 and 18 and a height of length 4. In terms of $\pi$, what is the area of the circle passing through all four vertices of the trapezoid?\vspace{3cm}
\item In how many ways can 2025 be written as a sum of 1s, 2s, and 3s? The order of the summands does not matter and not all numbers must appear in a given sum, so for instance, $2 + 2 + 1 + 1$ is a valid way to write $6$ as a sum and it is considered to be the same as $1 + 2 + 1 + 2$.\vspace{3cm}
\item For any positive integer $n$, let $\varphi(n)$ denote the number of positive integers less than or equal to $n$ which are relatively prime to $n$. Compute the sum of all fractions of the form $\varphi(k)/k^2$ that have terminating base fourteen representations. Express your answer as a common fraction.
\end{enumerate}

\end{document}