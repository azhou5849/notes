\documentclass{article}
\usepackage{mathtools,amssymb}
\usepackage{float,graphicx}
\usepackage{nopageno}
\usepackage[letterpaper]{geometry}

\setlength{\parindent}{0in}


\begin{document}

\section*{The Multiplication Principle}

\begin{enumerate}
\item Josh has an orange hat, a blue hat, and a green hat. He has a blue shirt, a green shirt, a red shirt, and a magenta shirt. He also has a pair of red pants and a pair of blue pants. How many different outfits can Josh make that consist of one hat, one shirt, and one pair of pants?\vspace{3cm}
\item How many odd five-digit counting numbers can be formed by choosing digits from the set\par \{1, 2, 3, 4, 5, 6, 7\} if digits \underline{can} be repeated?\vspace{3cm}
\item (2011 National Sprint Problem \#2) A local restaurant boasts that they have 240 different dinner combinations. A dinner combination consists of an appetizer, entree, and dessert. If the restaurant offers 10 appetizer choices and 6 entree choices, how many different dessert choices does it have?\vspace{3cm}
\item A sandwich restaurant offers 6 different meats and 5 different vegetables. For each sandwich, customers can choose at most one meat and up to 5 vegetables. How many different sandwiches does the restaurant offer?
\end{enumerate}


\newpage

\section*{Counting Factors (Divisors)}

\textit{Throughout, only positive factors (divisors) are considered.}

\begin{enumerate}
\item How many factors does 3600 have?\vspace{2cm}
\item (2002 School Sprint Problem \#22) How many odd whole numbers are factors of 180?\vspace{2cm}
\item (2012 National Sprint Problem \#28) How many whole numbers $n$, such that $100\leq n\leq 1000$, have the same number of odd factors as even factors?\vspace{2cm}
\item (2006 State Team) Emma plays with her square unit tiles by arranging all of them into different shaped rectangular figures. (For example, a $5\times 7$ rectangle would use 35 tiles and would be considered the same rectangle as a $7\times 5$ rectangle.) Emma can form exactly ten different such rectangular figures that each use all of her tiles. What is the least number of tiles Emma could have?\vspace{2cm}
\end{enumerate}
\textbf{Additional:} For any $k$, the \emph{sum of $k$-th powers of divisors function}, $\sigma_k$, takes a positive integer $n$ and returns the sum of the $k$-th powers of all (positive) divisors of $n$. For example,
\begin{equation*}
\sigma_3(12) = 1^3 + 2^3 + 3^3 + 4^3 + 6^3 + 12^3 = 2044.
\end{equation*}
If $n = p_1^{e_1}p_2^{e_2}\cdots p_r^{e_r}$ is the prime factorization of $n$, then
\begin{equation*}
\boxed{\sigma_k(n) = (1 + p_1^k + p_1^{2k} + \cdots + p_1^{e_1k})(1 + p_2^k + p_2^{2k} + \cdots + p_1^{e_2k})\cdots (1 + p_r^k + p_r^{2k} + \cdots + p_r^{e_rk}).}
\end{equation*}
In our example, $12 = 2^2\cdot 3^1$, so $\sigma_3(12) = (1 + 2^3 + (2^2)^3)(1 + 3^3)$. The most relevant special cases are that $\sigma_0$ is the number-of-divisors function and $\sigma_1$ is the sum-of-divisors function.


\newpage

\section*{Extensions}
\vspace{1cm}
\begin{enumerate}
\item \underline{\hspace{3in}}\vspace{1cm}
\item \underline{\hspace{3in}}\vspace{1cm}
\item \underline{\hspace{3in}}\vspace{1cm}
\item \underline{\hspace{3in}}\vspace{1cm}
\item \underline{\hspace{3in}}\vspace{1cm}
\item \underline{\hspace{3in}}\vspace{1cm}
\item \underline{\hspace{3in}} (skippable, or feel free to be silly)\vspace{1cm}
\item \underline{\hspace{3in}}
\end{enumerate}


\newpage

\section*{Extra Problems}

\begin{enumerate}
\item A sequence of numbers has the property that the sum of the first $n$ terms is given by $n^3 + n + 1$. What is the 100th term of the sequence?\vspace{3cm}
\item An isosceles trapezoid has bases of lengths 6 and 18 and a height of length 4. In terms of $\pi$, what is the area of the circle passing through all four vertices of the trapezoid?\vspace{3cm}
\item In how many ways can 2025 be written as a sum of 1s, 2s, and 3s? The order of the summands does not matter and not all numbers must appear in a given sum, so for instance, $2 + 2 + 1 + 1$ is a valid way to write $6$ as a sum and it is considered to be the same as $1 + 2 + 1 + 2$.\vspace{3cm}
\item For any positive integer $n$, let $\varphi(n)$ denote the number of positive integers less than or equal to $n$ which are relatively prime to $n$. Compute the sum of all fractions of the form $\varphi(k)/k^2$ that have terminating base fourteen representations. Express your answer as a common fraction.
\end{enumerate}

\end{document}